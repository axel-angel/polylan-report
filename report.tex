\documentclass[a4paper,12pt]{article}

\usepackage[utf8x]{inputenc}
\usepackage[english]{babel}
\usepackage[T1]{fontenc}
\usepackage{lmodern}

\usepackage{amsmath}
\usepackage{amssymb}
\usepackage{geometry}
\usepackage{a4wide}
\usepackage{enumerate}
\usepackage{graphicx}
\usepackage{lastpage}

\usepackage{fancyhdr}
\pagestyle{fancy}
\fancyhead{}
\fancyfoot{}

\lhead{Étienne Ischer \& Axel Angel}
\rhead{Polylan infrastructure}
\chead{Documentation}
\rfoot{Page \thepage\ of \pageref{LastPage}}
\lfoot{\today}

\pdfinfo{
    /Author (Étienne Ischer)
    /Title (Polylan infrastructure)
    /Subject (Report)
    }

\begin{document}
\section{Introduction}
This document goal is to explain and document the choice made by Polylan for their LAN parties.
The last such event was the first and biggest never made in Switzerland.
Polylan 23 had nearly 800 players in the same room to play video games, exchange files and enjoy a great time while the part of the committee who designed the whole system had to fix problems as they came, and they sure came.
The event was usually organized for around 230 players in the Rolex Learning Center but they decided to move to the freshly-built Congress Center to welcome more players.
These choices introduced new challenges: first the available material and constraints were not the same, second they were more than thrice the people of their last party and that changes how the LAN network was designed to scale.

In this document we want to talk and argument about the initial design and how it had to be changed to accommodate these new variables.
We will talk about the hardware, the software, how these components interact together, scale in theory as well as what happened in practice.

% TODO: comments are here to help to guide the text, they are in the french but
% the document should be written in english :)

\subsection{Old architecture}
% * Architecture/design avant:
%     - Ce qui allait, ce qui n'allait pas, pourquoi?
%     - Qu'est-ce qui posait problème (grave?)
%     - Les outils utilisés, qu'est-ce qui était infaisable, qu'est-ce qui maquait?

\subsection{Overview of changes}
%   * Changement dans les grandes lignes, résumé
%   * Qu'est-ce qui prends le plus de temps, pourquoi, est-ce solvable?


\section{Servers}

\subsection{Server infrastructure}
%   * Infra serveurs:
%       - Combien de machines, combien de VM, quelles services par VM, quel hardware
%       - Hyperviseur pour VM vs alternative, comment deploy N VMs rapidement, outils
%       - Redondance entre serveurs pour les services, quels softwares
%       - Électricité, faites-vous attention à la consommation? Consommation totale (average)?

\subsection{Deployment}
%   * Deployment
%       - Quels outils pour le déploiement, comment (qui pull/push?), pourquoi pas une alternative? Salt/Ansible/Puppet
%       - Gestion des configs et états des machines (+comment maintenir les configs avec les mainteneurs de paquets)
%       - Durée de vie d'une machine, comment gérer les mise à jours (destroy/recreate?)
%       - Question sécurité, ex: comment empêcher des problèmes si on a un accès root avec salt sur toutes les machines? Salt a eu un problème de sécurité récemment.
%       - Gestion du développement au sein de PolyLan, (ex: git), versionné, archivé, backups?
%       - Quel temps faut-il pour deploy ces services, si c'est long, est-ce problématique? Est-ce solvable?

\subsection{Monitoring}
%   * Monitoring
%       - Monitoring: SNMP, quels serveurs pull/push, quelles sont les data, pourquoi SNMP vs alternative, outils (Observium, est-ce que le real time apporte beaucoup?)
%       - Monitoring: quelles sont les informations les plus importantes? Pourquoi?
%       - Log: quels outils pour surveiller les logs et pourquoi est-ce utile? Real-time utile?
%       - Autres outils, ex: utilités des graphs, data en brute? Détection d'anormalité, statistiques?


\section{Switches and routers}
%- Routeurs/Switches

\subsection{Infrastructure}
%   * Infra
%       - Réseau en arbre (graph/schéma), qu'est-ce qui est relié à quoi (copper/fiber), comment nommer, pourquoi scale ainsi, quelle capacité sur les links? Testé en pratique?
%       - Routeurs / switches, où et quel hardware/software?
%       - Protocole STP, etc? Cisco intelligent? Éviter les boucles (loop-free?)? Hardware mixte, autre chose que Cisco?
%       - Hardware cisco/autre: quelque chose à dire?
%       - Configuration cisco: quelles sont les points importants?
%       - Comment tester avant comment ça scale'ra, est-ce concluant?

\subsection{Subnets and NAT}
%   * Subnets/NAT
%       - Quels sous-réseaux, comment relier, redondance?
%       - Comment organiser avec les joueurs répartis sur plusieurs switches
%       - Comment dupliquer des links pour que ça résiste aux links qui tombent, load-balancing, Cisco VSS
%       - Pourquoi VSS a foiré (firmware?)? Comment le détecter et le prévenir? Ou faire autrement (standalone)?
%       - Comment gérer avec le NAT? Cisco NAT table en hardware (quelle taille et capacité?), comment scale avec autant de joueurs, comment éviter que ça pète avec trop de connexions (rebooter n'est pas une solution)?
%       - Réseau internet externe (du bâtiment), combien d'ip, quel débit pour chaque link, comment load-balance (map avec les  subnets?) HTTP-proxy, QoS, priority queues?, qu'est-ce qui serait plus optimal (ex: pas de NAT/IPv6?)?
%       - Quelques stats sur la bande passante interne (globaux, intra-switchs, inter-switchs, overhead?)

\subsection{Authentification}
%   * Authentification
%       - Radius, ce qui avait avant (comme hack) pourquoi ça scale pas/pourquoi on a besoin de logs de connexion (légalité)
%       - Autre solution que Radius? Comment ça scale, comment load-balance?

\subsection{Software used}
%   * Archi software
%       - Reverse-Proxy: quels services sont derrière un proxy? Quels soft comme proxy (load-balance), ex: haproxy, etc?
%       - DHCP: comment donner des IPs (subnets?), les traquer, accepter dans les firewalls, etc
%       - DNS: comment scale à l'intérieur
%       - Web: nginx, autre? Comment scale avec Python/django?
%       - Database: postgres, comment scale? memcached, etc?
%       - Firewall/routeurs software: iptables/shorewall, etc? Comment gérer avec les firewall/routeurs hardware
%       - Comment détecter le DOS, le trafic inutile ou pertubateur? Et le bloquer, par qui, où?
%       - Autres services: serveurs de jeux, FTP, DC(++), etc


\section{What went wrong and how to improve}
%- Améliorations générales
%   * Problèmes rencontrés en prod mais pas en test?
%   * Problème hardware: cables, switches cassés, disque durs, etc?
%   * Réseau, comment rendre plus robuste? Autre hardware/software?
%   * D'autres facteurs externes qui empêchent de scale (manque de matos, d'ip internet, débit, etc?)
%   * Besoin de plus de personnes dans certains domaines? Réseau, pas assez d'expert? etc
%   * Autres…


\end{document}
